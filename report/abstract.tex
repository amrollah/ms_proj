\begin{abstract}
Managing fluctuation in photo-voltaic power plants which is frequent in cloudy days, is one of the big challenges that need to be solved in order to significantly increase its penetration into the power grid. One possible approach to predict short term variations is vision-based which includes a fish eye camera pointing into the sky, taking image sequences. Cloud states are predicted for near future using cloud segmentation and optical flow. In this context, we investigate the role of the cloudiness in shielding the direct sun irradiation and also reflecting the sunlight which increases the diffuse irradiation. Specifically, by analyzing our irradiation sensor measurements and image features we learn a soft sensor regressor for inferring irradiance of a given image. Direct component of irradiation is inferred using sun detection algorithm and cloud map. Finally the diffuse irradiation is estimated using several features derived from cloud state and date and time of image. Several regression algorithms are compared and Support Vector Ordinal Regression Machine delivers the best result with $43 W/m^2$ error and $36 W/m^2$  error standard deviation.
\end{abstract}
