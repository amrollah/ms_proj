\begin{abstract}
Managing short-term cloud-induced fluctuations in power output of photo-voltaic power plants is one of the main challenges that need to be solved in order to facilitate solar energy integration into the power grid.
Vision-based methods based on sky imagery for predicting future state of the sky in intra-hour horizons has been an active research topic in recent years to address this problem. However, modeling the complex effect of clouds on irradiance components specifically diffuse irradiance (DHI) which usually requires expensive measurement instruments is still far from a solved problem.
In this study, we use measurements of two irradiance sensors, one horizontal and one tilted towards north to create an inexpensive soft sensor for DHI and relate that to cloud state in sky images.  We build on top of a cloud segmentation algorithm to approximate direct irradiance (DNI). Then by removing the effect of DNI from titled sensor observations, we obtain DHI. After evaluating several non-image and image-based  features on different regression algorithms to estimate DHI, we show that using both of the feature types together improves the estimation accuracy by 40\% compared to using only non-based features.
Among the evaluated regression methods, K-nearest-neighbor delivers the best result with RMSE of $34.8 W/m^2$.
\end{abstract}
