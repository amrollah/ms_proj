\chapter{Conclusion}
\label{sec:conclusion_chapter}
Following the global movement for increasing the share of green energy sources in energy market, photovoltaic (PV) power plants gained major attention. However, the short-time cloud-induced fluctuations in the output power of a PV which are largely unpredicted, makes the integration of solar energy into electrical grid difficult and risky. In the recent decade, many researches aimed to predict these short-time changes using sky imagery. Complexity of effect of clouds on direct (DNI) and diffuse (DHI) irradiation components brings a lot of challenges. These works usually determine the cloud map in the sky and by finding the cloud motion vectors try to forecast cloud map in intra-hour time horizons. Using such cloud maps and the characteristics of images, these methods try to estimate total irradiation (GHI) which will hit the PV plant in those times which is highly correlated with the generated power. However, most of them require advanced and expensive irradiation measurement instruments.


This work specifically is concerned about estimating diffuse (non-direct) component of irradiance from sky images. We use data of two irradiance sensors, one horizontal and one tilted towards north, as well as a fisheye lens camera images of sky. A simplified binary approach is used for estimating DNI based on sun-state (i.e.either sun shining or sun not visible), meaning that the value is either equal to clear-sky DNI or zero. By removing the effect of DNI from titled sensor we get a soft sensor for DHI. We build on top of already existing cloud segmentation work at ABB to extract meaningful features from sky images and related them to DHI. We evaluated several non-image and image-based features to find the best performing feature set for predicting DHI. As the learning algorithm, the results of Linear Regression, K-nearest-neighbor (K-NN) and Support Vector Regression methods are compared. The results show K-NN outperforms other two methods and also that including both clear-sky DHI and cloud coverage features for regression improves the estimation accuracy by 40\% compared to using only non-image features. The best achieved RMSE is about 34.8 $W/m^2$.


For future work, one needs to extend our binary DNI model to include other more complicated sun-states as well. Using cloud classification as a feature vector and also for estimating cloud-base-height is another area to work on.