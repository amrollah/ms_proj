\chapter{Estimating Diffuse Horizontal Irradiance (DHI) from sky image}
In this chapter, a new approach that we developed for estimating DHI from sky images is explained. First, our experimental setup and data is presented, then we talk about clear-sky model used here, and why estimating DHI is very important in predicting GHI. Furthermore, DHI estimation using the irradiance sensors and also sky-images is discussed. Afterwards, machine learning regression methods for obtaining DHI from sky-image are studies. Finally, the general strategy for power prediction in a photovoltaic power plant using the forecast irradiance components is proposed.

\subsection{Experimental Setup}
This study is conducted on one of the photovoltaic(solar) power plants operated by ABB company. This PV plant which is located in Cavriglia region in Italy is chosen as a pilot site for the "forecasting power prediction using sky-imagery" project. Therefore, it is equipped with the following instruments for recording irradiation and sky images:
\begin{itemize}
\item A customized wide-angle high resolution (4MP) camera system with a fisheye lens covering 185 degrees of field of view. The camera is in a packaging attached to a pole on rooftop of the building next to the site. Figure \ref{fig:camera_on_site} shows the camera and its position next to the PV plates.
\item Two GHI pyranometer (irradiance sensor); located close to the camera on rooftop. One of the sensors is horizontally facing sky, and the other one facing north with around 40 degrees angle to the horizontal plane. It's worth mentioning that the PV plates are titled to south with a fixed angle around 30 degrees to get more sun exposure.  Th sensors are depicted in Figure \ref{fig:pyranometers}.
\item A thermometer for recording the temperature at the site.
\item A PC which is connected to the camera, pyranometers and thermometer via their software interfaces in order to configure sample rates and store taken images and irradiance measurements. The data of power generated by the PV plant is also sampled and stored for every day. All the data recorded during each day is been automatically transfered to the company samba sever at midnight using a control software running on the PC.
\end{itemize}

\begin{figure}[h]
\caption{wide-angle camera system used at Carviglia site}
\label{fig:camera_on_site}
\includegraphics[scale=.3]{camera_on_site}
\centering
\end{figure} 

\begin{figure}[h]
\caption{Two pyranometers (one horizontal, one 45 degrees titled to north) located close to the camera location}
\label{fig:pyranometers}
\includegraphics[scale=.3]{pyranometers}
\centering
\end{figure} 

\subsection{Acquired Data}
The camera system captures several images from the whole-sky every 8 seconds with different narrow exposure ranges. These images which are labeled according to capture time, are combined to create an HDR (High Dynamic Range) image for every sample time. The original narrow exposure images are generally deleted except the images at each hour time (i.e. around 7:00, 8:00, 9:00 etc.). Since capturing images at night is not useful for power prediction applications, camera is instructed to only take pictures during daytime (i.e. sunrise to sunset) which is obtained for that specific location for each day using mathematical models. Nevertheless, according to the captures images, this is not enough and still there are some black images taken at the minutes before sunrise and after sunset. Therefore, while processing the images on the application side, we exclude those images using a threshold on average pixel intensity of each image. This threshold is determined empirically. The images are further filtered against a sky mask which is been designed to exclude nearby mountains and buildings in the field of view and also limit the field of view to 170 degrees since transforming the points which are further in horizon is not accurate enough and the sun light is not negligible when the sun is in those points.  It is worth mentioning that using HDR images in the image processing step is one of the key advantages of this study to related works specifically \cite{tSchmidt15}.

The pyranometers measure GHI values with the sample rate of 6 seconds. The temperature is also recorded with the  same sample rate. However, the generate power is measured and logged every 3 seconds. These different sample rate make it necessary to interpolate the available data to find the estimated data for a time which there is no data available. Therefore, we can assign total irradiance, temperature and generated power to any given image using its capture time. This data acquisition setup has been running since 7th July 2015 until present. However, there are some short periods of time (usually lasting several days up to two weeks) which one of the sensors (pyranometers or the camera) had problems or the power plant was not working to produce power data. Considering the relatively small sample rate (less than 8 seconds), the amount of recorded data is big enough to make those off-days negligible in data processing steps. The data used in this study spans from 15th July to 10 February, meaning that many summer, autumn and winter days are available in the dataset to make it a good representation for the whole year. 


\subsection{Sun positions and sun states in image}
Knowing position of sun is very important both in cloud segmentation and in irradiance estimation. First of all, since our images are from a wide-angle fisheye lens, they need to be transformed into geometric coordinates by an un-distortion algorithm to make them ready for further image processing steps including sun position, cloud segmentation and etc. This can be done by multiplying the raw image coordinates to camera transformation matrix which consists of intrinsic and extrinsic parameters of the camera system. As described in section \ref{sec:image_undistortion_schmidt}, intrinsic parameters are calculated using image of a chessboard\cite{fisheye_undistort} and extrinsic parameters are estimated using Kabsch algorithm\cite{Kabsch_alg} based on position of the sun appeared in the image versus the expected position of sun in image. The theoretical sun positions are calculated for the location of our PV plant site at every image time-stamp using NREL algorithm \cite{our_sun_position} in Matlab represented in unit sphere polar coordinates. As shown in Figure \ref{fig:sun_position_angles} this position is described as two angles (zenith and azimuth) which are converted to Cartesian coordinates using sphere to Cartesian conversion and later on are scaled to the image size to correspond with an image pixel. That pixel will be assumed as center of the sun.

\begin{figure}[h]
\caption{Sun position angles.  source:\cite{sun_angle_pic}}
\label{fig:sun_position_angles}
\includegraphics[scale=.5]{sun_angles}
\centering
\end{figure}

In cloud segmentation, which is not the focus of this study, sun position is used to treat pixels close to sun according to different threshold than other pixels. Furthermore, a sun state detection inspect the area around sun position to classify sun state in the image into following 4 categories:
\begin{itemize}
\item sun\_flag=4: indicating the sun is visible in the image and it appears as a star shape emitting 6 symmetric strong rays.
\item sun\_flag=3: indicating the sun is visible in the image and it appears as a star shape but with 5 or less symmetric strong rays.
\item  sun\_flag=2: indicating the sun is visible in the image but it does not appear as a star shape. Instead, it appears as a small black dot with no strong rays.
\item sun\_flag=1: indicating the sun is not visible in the image and it is either covered by clouds or the sun position is out of field of view in the image.
\item  sun\_flag=-1: indicating there is an unexpected situation around sun position, for example star shape sun is detected far away from expected sun position which could be because of strange cloud formations.
\end{itemize}
One sample for each one of these categories is depicted in Figure \ref{fig:sun_states}. We are able distinguish between this states thanks to HDR images, otherwise this fine classification would not be possible.

\begin{figure}[h]
\caption{Different sun states, from left to right: sun flag=4, 3, 2, 1 respectively.}
\label{fig:sun_states}
\includegraphics[scale=.5]{sun_flag_4}
\includegraphics[scale=.5]{sun_flag_3}
\includegraphics[scale=.5]{sun_flag_2}
\includegraphics[scale=.5]{sun_flag_1}
\centering
\end{figure}

The variation of sun color in the images is so much that using Support Vector Machine for detecting sun is works very poorly. This issue is visible in Figure \ref{fig:sun_variation} which shows some clear sun samples from the images. Therefore, another approach using gray-scale images and geometrical symmetry detection is employed.

\begin{figure}[h]
\caption{Variation of sun appearance in the images.}
\label{fig:sun_variation}
\includegraphics[scale=.7]{suns}
\centering
\end{figure}

Using these sun states along with the sun position in cloud segmentation algorithm, will lead to better segmentation results close to sun that is particularly a difficult area for cloud segmentation due to highly saturated pixels with different sky or cloud colors. In irradiance estimation which is the main focus of this study, sun position is used to create two specific feature vectors in a bounding circle around the sun. These features are explained section \ref{sec:img-features} in detail.

\subsection{Clear-sky irradiance model}
Before dealing with the problem of irradiance estimation from cloudy images we should know first know how much the irradiance would be at any time in clear sky conditions when there is no cloud in the sky at all. Fortunately, there are a few number of models which provide irradiance components at each given time for many locations on the Earth. In this study we investigate two of the these methods, Ineichen \cite{clear_sky_model} and McClear \cite{mcclear_alg}. Both of these methods use location coordinates (latitude, longitude) and query time (consisting of year,day,month,day,minute,second) to calculate sun angles internally and return the irradiation based on optic formulas which determine how much sun should reach the ground in as direct sunlight and how much should be scattered in hemisphere and forms the diffuse irradiance. The amount of scattered sunlight is varying throughout the year and also depends on the location, ground albedo\footnote{The fraction of solar energy (shortwave radiation) which is reflected from the Earth back into space. It is a measure of the reflectivity of the earth's surface. Ice, especially with snow on top of it, has a high albedo.} and aerosol parameters including pressure, ozone column content, water vapour column content,  optical depth and Angstrom coefficient. 

\subsubsection{Ineichen method}
Ineichen model statistically and physically relates some irradiance measurements to the aforementioned parameters as Linke Turbidity profiles which are available for different locations . We used the default Linke Turbidity values which come as a separate file in PV-Lib toolbox \cite{pv_lib_toolbox} in Matlab and is representative for most locations in Europe. However, one might need to use other appropriate Linke Turbidity profiles for other locations to get more accurate results.
 
\subsubsection{McClear method}
On contrary to this approach, McClear uses a fully physical model that exploits recent aerosol properties, total column content in water vapour and ozone produced by the MACC project (Monitoring Atmosphere Composition and Climate). The MACC project, funded by the European Commission, uses  data of many Numerical Weather Prediction (NWP) centers distributed around the world to provide a global aerosol property forecasts together with physically consistent total column content in water vapour and ozone. In other words, since McClear uses synthetic data of NWP's, it does not depend on any local atmospheric observations for irradiance prediction. For the sake of speed, McClear irradiance estimates are pre-computed for the the location of these measurement centers and are interpolated for all other point on th Earth using a look-up table approach. Of course the closer we are to one of these measurement centers, the more accurate McClear estimate will be. The McClear irradiance estimates are available worldwide for every minutes from 2004 to present with 2 days lag under this web service \cite{mcclear_site}. This means that if we want to get the irradiance for today, we need to interpolate data of several days or weeks before to get an estimate for the current time. Nevertheless, one can use the original data of past 2 days for current time as well, since irradiance does not change considerably in two days.

\subsubsection{Comparison of simulation results}
To evaluate performance of these two clear-sky models we choose several days which have a significant clear part during the day (just because complete clear days are very rare). The chosen days should be from different months of year in order to represent performance of models for the whole year better. After observing the irradiance logs and images for verification, the following days were selected for comparison: 2015/07/19, 2015/08/03, 2015/09/21, 2015/10/24, 2015/11/24 and 2016/02/05. As an example, simulated GHI of three days are plotted in Figure \ref{fig:mcclear_vs_Ineichen_days} next to the observation. 

\begin{figure}[h]
\caption{GHI estimation of McClear vs Ineichen vs observation}
\label{fig:mcclear_vs_Ineichen_days}
\includegraphics[scale=.2]{2015_07_19_irr}
\includegraphics[scale=.2]{2015_10_24_irr}
\includegraphics[scale=.2]{2016_02_05_irr}
\centering
\end{figure}

As it can be seen, both methods can predict the shape of irradiance curve very accurately around the year, but both have biases in the result such that the simulated values are always smaller than observed ones. During the summer days, McClear and  Ineichen results are biased almost equally, and as we get closer to winter days, the bias of McClear gets smaller and bias of Ineichen get slightly bigger. 
Figure \ref{fig:mcclear_to_Ineichen} shows the correlation of simulated GHI of both models plotted with respect to the actual measurements of all of the examined days.

\begin{figure}[h]
\caption{Correlation of GHI simulation of McClear and Ineichen with observations on selected days}
\label{fig:mcclear_to_Ineichen}
\includegraphics[scale=.7]{mcclear_inceichen}
\centering
\end{figure}

This plot again shows that the bias of McClear and Ineichen for large values of GHI is the same, and for small GHI values McClear result is closer to the observed irradiance. however, it also suggests that since Ineichen methods has lower variation in terms of bias, this bias can be compensated with a scaling factor more easily than the bias of McClear which shows larger variation throughout the year. Note that, the outliers in this figure are representing cloudy times. Furthermore, this bias of Ineichen method is strongly related to the Turbidity factors that are neglected in our case by using the default values. It would be not surprising to see smaller bias if one uses Turbidity factors which are verified for the location of PV plant. The correlation of DNI and DHI values of both methods are illustrated in Figure \ref{fig:mcclear_vs_Ineichen_DNI_DHI}. One can see that DNI is predicted much higher in Ineichen method and DHI is also simulated slightly higher than McClear values. This behavior is intensified during autumn for DNI, but it is not varying a lot for DHI. Since we do not have observation values of DHI and DNI, we cannot compare correlation of methods' results to the observed values in this figure.  However, we can hypothesize than during a complete sunny day, if at a very short time (i.e. seconds) a cloud covers the sun completely, the direct irradiance (DHI) is almost zero and all the irradiation only comes from DHI source which is the scattered light in the sky. Therefore, we can use our GHI irradiance observations as DHI and compare it to the DHI simulated values for that particular time. 


\begin{figure}[h]
\caption{DNI and DHI correlation of McClear to Ineichen}
\label{fig:mcclear_vs_Ineichen_DNI_DHI}
\includegraphics[scale=.4]{mcclear_inceichen_DHI}
\includegraphics[scale=.4]{mcclear_inceichen_DNI}
\centering
\end{figure}

In Figure \ref{fig:McClear_DHI_vs_ineichen} this has been shown for a moment around 13:00 which sun is occluded by a small thick cloud and therefore, GHI is dropped rapidly to a value close to 150 which is very close to simulated DHI values from McClear and Ineichen methods. Also, there is no other cloud in the sky to influence the DHI component. Furthermore, the simulated DHI values are very close to each other at any time and resemble the shape of GHI relatively well during whole day. Thus, we can conclude that these models can predict DHI with good accuracy, and since DNI is a function of GHI and DHI according to Eq \ref{eq:irr_components}, DNI values calculated from McClear and Ineichen models are also accurate enough and for our application. This hypothesize has been verified by looking at many other data points where GHI observation gets very close to DHI simulation values and there is a cloud obstructing the direct sun light.

\begin{figure}[h]
\caption{Comparison of DHI of McClear and Ineichen to observed GHI}
\label{fig:McClear_DHI_vs_ineichen}
\includegraphics[scale=.5]{McClear_DHI_vs_ineichen}
\centering
\end{figure}

We know  that DNI should be always larger than DHI during clear-sky condition. Looking at Figure \ref{fig:clean_sky_day_irr}, pattern of changes in the simulated values of DNI and DHI during a day confirms this condition too.

\begin{figure}[h]
\caption{Variation of irradiance components during day}
\label{fig:clean_sky_day_irr}
\includegraphics[scale=.5]{clean_sky_day}
\centering
\end{figure}

\subsubsection{Choosing the clear-sky model}
As we discuss in this chapter, McClear and Ineichen models both predict the clear-sky irradiance components relatively accurately,however the bias for Ineichen method is more robust and manageable than McClear bias. Furthermore, obtaining the McClear values requires downloading the irradiance files from their web service since there is not offline library for calculating them. All this considered together, we decided to use Ineichen as the clear-sky irradiance model for this study. The scaling factor for compensating the bias in Ineichen is set to 1.08 which is empirically calculated based on several clear day observations.

\subsection{Estimating diffuse from pyranometers}

Using clear-sun-flag and clear-sky DNI for calculating diffuse from tilted plate
 
ii.	clear-sky DNI is a good approximation for actual DNI. Showing some clear or cloudy days to prove this point.

We only consider images with complete visible sun or no sun at all.
ii.	Comparing tilted diffuse with diffuse from the main irradiance plate, (then correcting tilted diffuse??). Showing its robustness.
Show the result of relation of DNI DHI to irr1 irr2 with plots.my  recent work.

\subsection{Key image features affecting DHI}
\label{sec:img-features}
iii.	Investigate different parameters which affect diffuse
\subsubsection{Discuss cloud coverage geometrical polar feature around sun and also the whole image}
\subsubsection{Discuss saturation detection algorithm and its results}


iv.	Show their correlation to diffuse, discuss cases based on images and corresponding irradiance components


\subsection{Learning the relation between image features and DHI}
Regression, or svorim method for estimation
Compare to other method to regress, or penalty [no result]



\subsection{Translating irradiance to power}
\subsubsection{Estimating shadow ratio on the plant site}
refer to Figure \ref{fig:overall_system_design_power}.
Projecting plant coordinates into the sky using assumed cloud height
Adaptation to smooth changes of power by using the histogram of irradiance over past 30 minutes
\begin{figure}[h]
\caption{Overall simple system design for power prediction}
\label{fig:overall_system_design_power}
\includegraphics[scale=.5]{Overall_Simple_System_Design}
\centering
\end{figure}
