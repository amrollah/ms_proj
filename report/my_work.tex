\chapter{Estimating Diffuse Horizontal Irradiance (DHI) from sky image}
In this chapter, a new approach that we developed for estimating DHI from sky images is explained. First, our experimental setup and data is presented, then we talk about clear-sky model used here, and why estimating DHI is very important in predicting GHI. Furthermore, DHI estimation using the irradiance sensors and also sky-images is discussed. Afterwards, machine learning regression methods for obtaining DHI from sky-image are studies. Finally, the general strategy for power prediction in a photovoltaic power plant using the forecast irradiance components is proposed.

\subsection{Experimental Setup}
This study is conducted on one of the photovoltaic(solar) power plants operated by ABB company. This PV plant which is located in Cavriglia region in Italy is chosen as a pilot site for the "forecasting power prediction using sky-imagery" project. Therefore, it is equipped with the following instruments for recording irradiation and sky images:
\begin{itemize}
\item A customized high resolution (4MP) camera system with a fisheye lens covering 185 degrees of field of view. The camera is in a packaging attached to a pole on rooftop of the building next to the site. Figure shows the camera and its position next to the PV plates.
\item Two GHI pyranometer (irradiance sensor); located close to the camera on rooftop. One of the sensors is horizontally facing sky, and the other one facing north with around 40 degrees angle to the horizontal plane. Th sensors are depicted in Figure.
\item A thermometer for recording the temperature at the site.
\item A PC which is connected to the camera, pyranometers and thermometer via their software interfaces in order to configure sample rates and store taken images and irradiance measurements. The data of power generated by the PV plant is also sampled and stored for every day. All the data recorded during each day is been automatically transfered to the company samba sever at midnight using a control software running on the PC.
\end{itemize}


\subsection{Acquired Data}
The camera system captures several images from the whole-sky every 8 seconds with different narrow exposure ranges. These images which are labeled according to capture time, are combined to create an HDR (High Dynamic Range) image for every sample time. The original narrow exposure images are generally deleted except the images at each hour time (i.e. around 7:00, 8:00, 9:00 etc.). Since capturing images at night is not useful for power prediction applications, camera is instructed to only take pictures during daytime (i.e. sunrise to sunset) which is obtained for that specific location for each day using available models. Using HDR images in the image processing step is one of the key differences between this study and related works specifically \cite{tSchmidt15}.

\paragraph
The pyranometers measure GHI values with the sample rate of 6 seconds. The temperature is also recorded with the  same sample rate. However, the generate power is measured and logged every 3 seconds. These different sample rate make it necessary to interpolate the available data to find the estimated data for a time which there is no data available. Therefore, we can assign total irradiance, temperature and generated power to any given image using its capture time. This data acquisition setup has been running since 7th July 2015 until present. However, there are some short periods of time (usually lasting several days up to two weeks) which one of the sensors (pyranometers or the camera) had problems or the power plant was not working to produce power data. Considering the relatively small sample rate (less than 8 seconds), the amount of recorded data is big enough to make those off-days negligible in data processing steps. The data used in this study spans from 15th July to 10 February, meaning that many summer, autumn and winter days are available in the dataset to make it a good representation for the whole year.


\subsection{Camera calibration for sun positions}



\subsection{Clear-sky irradiance model}
i.	Comparison of McClear model vs ineichen vs our measurements
ii.	clear-sky DNI is a good approximation for actual DNI. Showing some clear or cloudy days to prove this point.


\subsection{Estimating diffuse from pyranometers}
Using clearsun-flag and clear-sky DNI for calculating diffuse from tilted plate
We only consider images with complete visible sun or no sun at all.
ii.	Comparing tilted diffuse with diffuse from the main irradiance plate, then correcting tilted diffuse. Showing its robustness.
Show the result of relation of DNI DHI to irr1 irr2 with plots.

\subsection{Key image features affecting DHI}
iii.	Investigate different parameters which affect diffuse
\subsubsection{Discuss cloud coverage geometrical polar feature around sun and also the whole image}
\subsubsection{Discuss saturation detection algorithm and its results}

iv.	Show their correlation to diffuse, discuss cases based on images and corresponding irradiance components


\subsection{Learning the relation between image features and DHI}
Regression, or svorim method for estimation
Compare to other method to regress, or penalty [no result]



\subsection{Translating irradiance to power}
\subsubsection{Estimating shadow ratio on the plant site}
Projecting plant coordinates into the sky using assumed cloud height
Adaptation to smooth changes of power by using the histogram of irradiance over past 30 minutes

